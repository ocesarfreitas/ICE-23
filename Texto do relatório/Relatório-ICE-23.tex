% Options for packages loaded elsewhere
\PassOptionsToPackage{unicode}{hyperref}
\PassOptionsToPackage{hyphens}{url}
%
\documentclass[
  12,
  dvipsnames]{article}
\title{Relatório Índice das Cidades Empreendedoras 2023}
\author{César Freitas \and Vítor Borges \and Arnaldo}
\date{September 06, 2022}

\usepackage{amsmath,amssymb}
\usepackage{lmodern}
\usepackage{iftex}
\ifPDFTeX
  \usepackage[T1]{fontenc}
  \usepackage[utf8]{inputenc}
  \usepackage{textcomp} % provide euro and other symbols
\else % if luatex or xetex
  \usepackage{unicode-math}
  \defaultfontfeatures{Scale=MatchLowercase}
  \defaultfontfeatures[\rmfamily]{Ligatures=TeX,Scale=1}
\fi
% Use upquote if available, for straight quotes in verbatim environments
\IfFileExists{upquote.sty}{\usepackage{upquote}}{}
\IfFileExists{microtype.sty}{% use microtype if available
  \usepackage[]{microtype}
  \UseMicrotypeSet[protrusion]{basicmath} % disable protrusion for tt fonts
}{}
\usepackage{xcolor}
\IfFileExists{xurl.sty}{\usepackage{xurl}}{} % add URL line breaks if available
\IfFileExists{bookmark.sty}{\usepackage{bookmark}}{\usepackage{hyperref}}
\hypersetup{
  pdftitle={Relatório Índice das Cidades Empreendedoras 2023},
  pdfauthor={César Freitas; Vítor Borges; Arnaldo},
  hidelinks,
  pdfcreator={LaTeX via pandoc}}
\urlstyle{same} % disable monospaced font for URLs
\usepackage[left=3cm, right=2cm, top=3cm, bottom=2cm]{geometry}
\usepackage{color}
\usepackage{fancyvrb}
\newcommand{\VerbBar}{|}
\newcommand{\VERB}{\Verb[commandchars=\\\{\}]}
\DefineVerbatimEnvironment{Highlighting}{Verbatim}{commandchars=\\\{\}}
% Add ',fontsize=\small' for more characters per line
\usepackage{framed}
\definecolor{shadecolor}{RGB}{248,248,248}
\newenvironment{Shaded}{\begin{snugshade}}{\end{snugshade}}
\newcommand{\AlertTok}[1]{\textcolor[rgb]{0.94,0.16,0.16}{#1}}
\newcommand{\AnnotationTok}[1]{\textcolor[rgb]{0.56,0.35,0.01}{\textbf{\textit{#1}}}}
\newcommand{\AttributeTok}[1]{\textcolor[rgb]{0.77,0.63,0.00}{#1}}
\newcommand{\BaseNTok}[1]{\textcolor[rgb]{0.00,0.00,0.81}{#1}}
\newcommand{\BuiltInTok}[1]{#1}
\newcommand{\CharTok}[1]{\textcolor[rgb]{0.31,0.60,0.02}{#1}}
\newcommand{\CommentTok}[1]{\textcolor[rgb]{0.56,0.35,0.01}{\textit{#1}}}
\newcommand{\CommentVarTok}[1]{\textcolor[rgb]{0.56,0.35,0.01}{\textbf{\textit{#1}}}}
\newcommand{\ConstantTok}[1]{\textcolor[rgb]{0.00,0.00,0.00}{#1}}
\newcommand{\ControlFlowTok}[1]{\textcolor[rgb]{0.13,0.29,0.53}{\textbf{#1}}}
\newcommand{\DataTypeTok}[1]{\textcolor[rgb]{0.13,0.29,0.53}{#1}}
\newcommand{\DecValTok}[1]{\textcolor[rgb]{0.00,0.00,0.81}{#1}}
\newcommand{\DocumentationTok}[1]{\textcolor[rgb]{0.56,0.35,0.01}{\textbf{\textit{#1}}}}
\newcommand{\ErrorTok}[1]{\textcolor[rgb]{0.64,0.00,0.00}{\textbf{#1}}}
\newcommand{\ExtensionTok}[1]{#1}
\newcommand{\FloatTok}[1]{\textcolor[rgb]{0.00,0.00,0.81}{#1}}
\newcommand{\FunctionTok}[1]{\textcolor[rgb]{0.00,0.00,0.00}{#1}}
\newcommand{\ImportTok}[1]{#1}
\newcommand{\InformationTok}[1]{\textcolor[rgb]{0.56,0.35,0.01}{\textbf{\textit{#1}}}}
\newcommand{\KeywordTok}[1]{\textcolor[rgb]{0.13,0.29,0.53}{\textbf{#1}}}
\newcommand{\NormalTok}[1]{#1}
\newcommand{\OperatorTok}[1]{\textcolor[rgb]{0.81,0.36,0.00}{\textbf{#1}}}
\newcommand{\OtherTok}[1]{\textcolor[rgb]{0.56,0.35,0.01}{#1}}
\newcommand{\PreprocessorTok}[1]{\textcolor[rgb]{0.56,0.35,0.01}{\textit{#1}}}
\newcommand{\RegionMarkerTok}[1]{#1}
\newcommand{\SpecialCharTok}[1]{\textcolor[rgb]{0.00,0.00,0.00}{#1}}
\newcommand{\SpecialStringTok}[1]{\textcolor[rgb]{0.31,0.60,0.02}{#1}}
\newcommand{\StringTok}[1]{\textcolor[rgb]{0.31,0.60,0.02}{#1}}
\newcommand{\VariableTok}[1]{\textcolor[rgb]{0.00,0.00,0.00}{#1}}
\newcommand{\VerbatimStringTok}[1]{\textcolor[rgb]{0.31,0.60,0.02}{#1}}
\newcommand{\WarningTok}[1]{\textcolor[rgb]{0.56,0.35,0.01}{\textbf{\textit{#1}}}}
\usepackage{graphicx}
\makeatletter
\def\maxwidth{\ifdim\Gin@nat@width>\linewidth\linewidth\else\Gin@nat@width\fi}
\def\maxheight{\ifdim\Gin@nat@height>\textheight\textheight\else\Gin@nat@height\fi}
\makeatother
% Scale images if necessary, so that they will not overflow the page
% margins by default, and it is still possible to overwrite the defaults
% using explicit options in \includegraphics[width, height, ...]{}
\setkeys{Gin}{width=\maxwidth,height=\maxheight,keepaspectratio}
% Set default figure placement to htbp
\makeatletter
\def\fps@figure{htbp}
\makeatother
\setlength{\emergencystretch}{3em} % prevent overfull lines
\providecommand{\tightlist}{%
  \setlength{\itemsep}{0pt}\setlength{\parskip}{0pt}}
\setcounter{secnumdepth}{5}
\usepackage[utf8]{inputenc}
\usepackage{dcolumn}
\usepackage{csquotes}
\usepackage{indentfirst}
\renewcommand{\abstractname}{Resumo}
\renewcommand{\tablename}{Tabela}
\renewcommand{\figurename}{Figura}
\usepackage{hyperref}
\usepackage{microtype}
\usepackage{bookmark}
\usepackage[onehalfspacing]{setspace}
\hypersetup{ colorlinks=true, linkcolor=black, filecolor=black, urlcolor=blue, citecolor=YellowOrange!95!black }
\usepackage{float}
\floatplacement{figure}{H}
\ifLuaTeX
  \usepackage{selnolig}  % disable illegal ligatures
\fi

\begin{document}
\maketitle
\begin{abstract}
A
\end{abstract}

\renewcommand*\contentsname{Sumário}
{
\setcounter{tocdepth}{2}
\tableofcontents
}
\newpage

\begin{Shaded}
\begin{Highlighting}[]
\ImportTok{import}\NormalTok{ pandas }\ImportTok{as}\NormalTok{ pd}
\ImportTok{import}\NormalTok{ numpy }\ImportTok{as}\NormalTok{ np}
\ImportTok{import}\NormalTok{ basedosdados }\ImportTok{as}\NormalTok{ bd}
\end{Highlighting}
\end{Shaded}

\hypertarget{introducao}{%
\section{Introducao}\label{introducao}}

\hypertarget{determinantes}{%
\section{Determinantes}\label{determinantes}}

\hypertarget{determinante-ambiente-regulatuxf3rio}{%
\subsection{Determinante Ambiente
Regulatório}\label{determinante-ambiente-regulatuxf3rio}}

\hypertarget{subdeterminante-tempo-de-processos}{%
\subsubsection{Subdeterminante Tempo de
Processos}\label{subdeterminante-tempo-de-processos}}

\hypertarget{apend}{%
\section*{Apêndice}\label{apend}}
\addcontentsline{toc}{section}{Apêndice}

\hypertarget{script_ind}{%
\subsection*{Scripts Indicadores}\label{script_ind}}
\addcontentsline{toc}{subsection}{Scripts Indicadores}

\hypertarget{amostra}{%
\subsubsection*{Amostra}\label{amostra}}
\addcontentsline{toc}{subsubsection}{Amostra}

\begin{Shaded}
\begin{Highlighting}[]
\CommentTok{\# Leitura da base}
\NormalTok{df }\OperatorTok{=}\NormalTok{ pd.read\_excel(}\StringTok{\textquotesingle{}POP2021\_20220711.xls\textquotesingle{}}\NormalTok{, }\StringTok{\textquotesingle{}Municípios\textquotesingle{}}\NormalTok{)}
\NormalTok{df }\OperatorTok{=}\NormalTok{ df.head(}\OperatorTok{{-}}\DecValTok{32}\NormalTok{)}
\NormalTok{df.columns }\OperatorTok{=}\NormalTok{ df.iloc[}\DecValTok{0}\NormalTok{]}
\NormalTok{df }\OperatorTok{=}\NormalTok{ df.drop(}\DecValTok{0}\NormalTok{)}
\NormalTok{df.index }\OperatorTok{=} \BuiltInTok{range}\NormalTok{(}\BuiltInTok{len}\NormalTok{(df))}
\ControlFlowTok{for}\NormalTok{ i }\KeywordTok{in} \BuiltInTok{range}\NormalTok{(}\BuiltInTok{len}\NormalTok{(df)):}
    \ControlFlowTok{if} \BuiltInTok{type}\NormalTok{(df.iloc[i][}\StringTok{\textquotesingle{}POPULAÇÃO ESTIMADA\textquotesingle{}}\NormalTok{]) }\OperatorTok{==} \BuiltInTok{str}\NormalTok{:}
\NormalTok{        pop }\OperatorTok{=} \BuiltInTok{int}\NormalTok{(df.iloc[i][}\StringTok{\textquotesingle{}POPULAÇÃO ESTIMADA\textquotesingle{}}\NormalTok{].split(}\StringTok{\textquotesingle{}(\textquotesingle{}}\NormalTok{)[}\DecValTok{0}\NormalTok{].replace(}\StringTok{\textquotesingle{}.\textquotesingle{}}\NormalTok{,}\StringTok{\textquotesingle{}\textquotesingle{}}\NormalTok{))}
\NormalTok{        df.at[i, }\StringTok{\textquotesingle{}POPULAÇÃO ESTIMADA\textquotesingle{}}\NormalTok{] }\OperatorTok{=}\NormalTok{ pop}
\NormalTok{top100 }\OperatorTok{=}\NormalTok{ df.sort\_values(by}\OperatorTok{=}\NormalTok{[}\StringTok{\textquotesingle{}POPULAÇÃO ESTIMADA\textquotesingle{}}\NormalTok{], ascending}\OperatorTok{=}\VariableTok{False}\NormalTok{).head(}\DecValTok{101}\NormalTok{)}

\CommentTok{\# Tratando a base}
\NormalTok{am }\OperatorTok{=}\NormalTok{ pd.read\_excel(}\StringTok{\textquotesingle{}Amostra.xlsx\textquotesingle{}}\NormalTok{)}
\ControlFlowTok{for}\NormalTok{ i }\KeywordTok{in} \BuiltInTok{range}\NormalTok{(}\BuiltInTok{len}\NormalTok{(am)):}
\NormalTok{    name }\OperatorTok{=}\NormalTok{ am[}\StringTok{\textquotesingle{}Município\textquotesingle{}}\NormalTok{][i].split(}\StringTok{\textquotesingle{}{-}\textquotesingle{}}\NormalTok{)[}\DecValTok{0}\NormalTok{].strip()}
\NormalTok{    am.at[i,}\StringTok{\textquotesingle{}Município\textquotesingle{}}\NormalTok{] }\OperatorTok{=}\NormalTok{ name}

\KeywordTok{def}\NormalTok{ Union(lst1, lst2):}
\NormalTok{    final\_list }\OperatorTok{=} \BuiltInTok{list}\NormalTok{(}\BuiltInTok{set}\NormalTok{(lst1) }\OperatorTok{|} \BuiltInTok{set}\NormalTok{(lst2))}
    \ControlFlowTok{return}\NormalTok{ final\_list}

\NormalTok{final\_sample }\OperatorTok{=}\NormalTok{ Union(am[}\StringTok{\textquotesingle{}Município\textquotesingle{}}\NormalTok{],top100[}\StringTok{\textquotesingle{}NOME DO MUNICÍPIO\textquotesingle{}}\NormalTok{])}

\ControlFlowTok{for}\NormalTok{ i }\KeywordTok{in}\NormalTok{ am[}\StringTok{\textquotesingle{}Município\textquotesingle{}}\NormalTok{]:}
    \ControlFlowTok{if}\NormalTok{ i }\KeywordTok{not} \KeywordTok{in} \BuiltInTok{list}\NormalTok{(top100[}\StringTok{\textquotesingle{}NOME DO MUNICÍPIO\textquotesingle{}}\NormalTok{]):}
        \BuiltInTok{print}\NormalTok{(i)}

\CommentTok{\# Criando o arquivo}
\NormalTok{top100.to\_csv(}\StringTok{\textquotesingle{}100{-}municipios.csv\textquotesingle{}}\NormalTok{, index}\OperatorTok{=}\VariableTok{False}\NormalTok{)}
\end{Highlighting}
\end{Shaded}


\end{document}
